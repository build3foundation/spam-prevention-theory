\newpage
\section{The Cost of Yes}
Like in determining the route for a UPS driver, the non-optimal solution is
often better than the optimal solution which would take excessive
computational power.

In our world, computational power is the length of time it takes an engineer
to find the optimal solution. Our ‘free’ services for CA should include that
which takes the least computational effort to find the first legal solution.
In order to gain access to more processing power, the requester must pay the
network fee (ie pre-paid CA services).

While we don’t have a deterministic or stochastic approach yet, it does provide inspiration for ways to implement a calculus for making decisions on handling a request that may be outside of scope.

One way to determine the computational requirements is to evaluate them along with what litigious and economic risk exposure would occur if the request was declined.

If the risk is below a threshold, then we would either outright decline the
request or send the client anestimate of the effor to be pre-paid through the
\href{https://www.permitzip.com/ca-requests}{pre-paid CA portal}.

\subsection{example math}

I'm going to write$E=mc^2$

Now I want to have math not in a sentence like this $$E=mc^2$$ you know!

% Using & on either size of equal signs aligns the equations along the equal signs
\begin{eqnarray}
    -\frac{\hbar^2}{2m}\frac{d^2\Psi}{dx^2}&=&E\Psi \\
    d&=&v_{word}t+\frac{1}{2}\cdot at^2 \label{eq:example_one} \\
    \left(\frac{1}{2}\right)\cdot 2 &=& 1 \\
\end{eqnarray}

$$\left|-7 \right| = 7$$

\begin{equation}
    x^{2^y}
\end{equation}

$$ \sqrt{5} \neg 29 $$

\begin{equation} \label{eq:example_two}
    \pi \times \sqrt{4}
\end{equation}


The first equation is equation \ref{eq:example_one} so that's it. But also
there is another equation \ref{eq:example_two} which is so neat.

The introduction is found on page \pageref{sec:intro}.

% A table
\begin{table}[H]
    \centering
    \caption[This is optional caption, without reference]{Local caption, with
        reference \cite{ricardo}
    }
    \begin{tabular}{l c r} % left, center, right aligned headings
        Date     & In tree? & Raining? \\ \hline
        April 26 & Yes      & Yes      \\
        June 7   & Yes      & No       \\
        June 20  & Yes      & No       \\
    \end{tabular}
    \label{tab:example_table}

\end{table}

Table \ref{tab:example_table} logs times things happened.

\subsection{Here is a subsection}
This is a sub section but it references this other thing!

\subsubsection{another one}
This is a sub subsection.

This sub subsection is used for containing a list!

% Lists
\begin{itemize}
    \item This is a first line
    \item This is a second one that's longer and text wraps and it's nice super
          nice i love it super nice!
          \begin{itemize}
              \item Another one
              \item more!
          \end{itemize}
    \item More
    \item Another more!
          \begin{enumerate}
              \item Oh
              \item here go the numbers
          \end{enumerate}
\end{itemize}
\cleardoublepage
